\documentclass[hidelinks,12pt]{article}

\usepackage[brazil]{babel}
\usepackage[utf8]{inputenc}
\usepackage{amsmath}
\usepackage{amsfonts}
\usepackage{amssymb}
\usepackage{indentfirst}
\usepackage{color}
\usepackage{mathrsfs}
\usepackage{pgfplots}
\usepackage{hyperref}
\usepackage{fancyhdr}
\usepackage{graphicx}
\usepackage[export]{adjustbox}
\newcommand{\icon}[1]{\includegraphics[height=12pt]{#1}}
\newcommand{\bigicon}[1]{\includegraphics[height=50pt]{#1}}

\newcommand{\iconb}[1]{\includegraphics[height=20pt]{#1}}
\setcounter{secnumdepth}{5}

\fancypagestyle{plain}{%
	\fancyfoot{}%
	\fancyhead{}%
}


\begin{document}
\pagenumbering{gobble}
\pagestyle{fancy}


\lhead{\bigicon{Figures/ufu}}
\chead{{\footnotesize UNIVERSIDADE FEDERAL DE UBERLÂNDIA \\ FACULDADE DE CIÊNCIA DA COMPUTAÇÃO \\ Inteligência Computacional} \\ \scriptsize{Av. João Naves de Ávila 2121, Campus Santa Mônica} }
\rhead{\bigicon{Figures/facom}}
\lfoot{}
\cfoot{}
\rfoot{}
\vspace*{10cm}
\begin{figure}[!h]
	\centering
	\Huge{Algoritmo das Formigas aplicado ao problema do caixeiro viajante}


\end{figure}


\newpage
\fancyhead[C]{}
\fancyhead[R]{}
\fancyhead[L]{\leftmark}
\fancyfoot{}
\fancyfoot[L]{{\footnotesize  Inteligência Computacional}}
\fancyfoot[C]{\hspace{1.5cm}\thepage}
\fancyfoot[R]{{\footnotesize Formigas}}
\pagenumbering{arabic}


\tableofcontents

{\let\thefootnote\relax\footnotetext{\textit{UFU, Universidade Federal de Uberlândia, Minas Gerais, Brasil}}}

\newpage

\section{Introdução}

	O relatório informa as resoluções dos exercícios pedidos no trabalho de formigas, bem como as abordagens utilizadas para chegar às respostas desejadas.

\section{Especificação}
	
	
	\subsection{Entendimento do problema}
	
		Entende-se que, quando a professora especificou que \emph{"O aluno deve executar 50 execuções do algoritmo em cada instância avaliada do PCV (...)"}, deve-se executar 50 vezes o processo de execução total do algoritmo dado um número determinado de mudanças no feromônio.
		
		A partir desse processo, será pego o melhor resultado da melhor formiga no fim de cada execução do algoritmo. Após isso , será selecionado o melhor resultado de todas as soluções.
		
	\subsection{Parâmetros}
		
		Os parâmetros que variam no problema são:
		\begin{enumerate}
			\item \textbf{Quantidade de mudanças na matriz de feromônio}: F
			\item \textbf{Feromônio excretado pela formiga}: Q
			\item \textbf{Taxa de evaporação do feromônio}: P
			\item \textbf{Semente utilizada na Roleta}: S (sempre será seed = 0)
			\item \textbf{Alfa}: A  - Quantificador que eleva a matriz de feromônio 
			\item \textbf{Beta}: B - Quantifador que eleva a matriz de distância 
		\end{enumerate} 
		
\section{Exercícios}

	Seguem as respostas para as tabelas 1, 2, 3 e 4 referentes ao trabalho.
	
	
\subsection{Exercício 1}
	
	\subsubsection{Valores dos Parâmetros}	
	Para o exercício 1, serão adotados os seguintes parâmetros:
	
	\begin{enumerate}
		\item \textbf{F}:  50
		\item \textbf{Q}:  500, 5000
		\item \textbf{P}:  0.5, 0.05
		\item \textbf{A}:  3
		\item \textbf{B}:  2
	
	\end{enumerate}
	
	
	
	\subsection{Dados das Execuções}
	\begin{table}[!h]
	\centering
		\begin{tabular}{|c|c|}
		\hline
		\multicolumn{1}{|c|}{Parâmetros (F, Q, P)} & \multicolumn{1}{|c|}{Custo do Melhor Caminho} \\ \hline
		50, 500, 0.5                               &         2164                                     \\ \hline
		50, 500, 0.05                              &         2011                                     \\ \hline
		50, 5000, 0.5                              &         2011                                     \\ \hline
		50, 5000, 0.05                             &         2011                                     \\ \hline
		\end{tabular}
	\end{table}
	
	\subsection{Melhor Caminho}
	
			\begin{figure}[!h]
				\centering
				\includegraphics[scale=0.7]{Figures/BP1.png}
			\end{figure}
				
		
		
\subsection{Exercício 2}
		
	
		\subsubsection{Valores dos Parâmetros}	
		Para o exercício 2, serão adotados os seguintes parâmetros:
		
		\begin{enumerate}
			\item \textbf{F}: 50
			\item \textbf{Q}: 700, 5000
			\item \textbf{P}: 0.3, 0.05
			\item \textbf{A}:  3
			\item \textbf{B}:  2		
		\end{enumerate}
		
		\subsection{Dados das Execuções}
		\begin{table}[!h]
		\centering
		
			\begin{tabular}{|c|c|}
			\hline
			\multicolumn{1}{|c|}{Parâmetros (F, Q, P)} & \multicolumn{1}{|c|}{Custo do Melhor Caminho} \\ \hline
			50, 700, 0.3                               &           8238                                   \\ \hline
			50, 700, 0.05                              &           7721                                  \\ \hline
			50, 5000, 0.3                              &           7918                                   \\ \hline
			50, 5000, 0.05                             &           8659                                   \\ \hline

			\end{tabular}
		\end{table}
		
		\subsection{Melhor Caminho}
		
			\begin{figure}[!h]
				\centering
				\includegraphics[scale=0.7]{Figures/BP2.png}
			\end{figure}
	
\subsection{Exercício 3}
		
	\subsubsection{Valores dos Parâmetros}	
	Para o exercício 3, serão adotados os seguintes parâmetros:
	
	\begin{enumerate}
		\item \textbf{F}: 50 ,70
		\item \textbf{Q}: 1500, 5000
		\item \textbf{P}: 0.3
		\item \textbf{A}:  2,3
		\item \textbf{B}:  2,3	
	\end{enumerate}
	
	\newpage
	\subsection{Dados das Execuções}
	\begin{table}[!h]
		\centering
		
			\begin{tabular}{|c|c|}
			\hline
			\multicolumn{1}{|c|}{Parâmetros (A, B, F, Q, P)} & \multicolumn{1}{|c|}{Custo do Melhor Caminho} \\ \hline
			3, 2, 50, 1500, 0.3                               &              39186                                \\ \hline

			3, 2, 50, 5000, 0.3                              &               38095                               \\ \hline
			
			3, 2, 70, 1500, 0.3                               &              39556                                \\ \hline
			
			3, 2, 70, 5000, 0.3                              &                40044                              \\ \hline
			
			2, 3, 50, 1500, 0.3                               &               37269                               \\ \hline
			
			2, 3, 50, 5000, 0.3                              &              38095                                \\ \hline
			
			2, 3, 70, 1500, 0.3                               &            38696                                  \\ \hline
			
			2, 3, 70, 5000, 0.3                              &             37978                                 \\ \hline
			
			\end{tabular}
	\end{table}
	
	\subsection{Melhor Caminho}
	
			\begin{figure}[!h]
				\centering
				\includegraphics[scale=0.7]{Figures/BP3.png}
			\end{figure}	
	
\subsection{Exercício 4}

	\subsubsection{Valores dos Parâmetros}	
	Para o exercício 4, serão adotados os seguintes parâmetros:
	
	\begin{enumerate}
		\item \textbf{F}: 50
		\item \textbf{Q}: 5000
		\item \textbf{P}: 0.05, 0,3
		\item \textbf{A}:  2,3
		\item \textbf{B}:  2,3	
	\end{enumerate}
	\newpage
	\subsection{Dados das Execuções}
	\begin{table}[!h]
		\centering
		\label{my-label}
			\begin{tabular}{|c|c|}
			\hline
			\multicolumn{1}{|c|}{Parâmetros (A, B, F, Q, P)} & \multicolumn{1}{|c|}{Custo do Melhor Caminho} \\ \hline
			3, 2, 50, 5000, 0.05                              &            10224                                  \\ \hline
			2, 3, 50, 5000, 0.05                             &              9158                                \\ \hline
			3, 3, 50, 5000, 0.05                              &            10152                                  \\ \hline
			2, 2, 50, 5000, 0.05                             &              10401                                \\ \hline
			3, 2, 50, 5000, 0.3                              &               11928                               \\ \hline
			2, 3, 50, 5000, 0.3                             &               9503                               \\ \hline
			3, 3, 50, 5000, 0.3                              &               10076                               \\ \hline
			2, 2, 50, 5000, 0.3                             &              9814                                \\ \hline
			\end{tabular}
	\end{table}

	\subsection{Melhor Caminho}
	
			\begin{figure}[!h]
				\centering
				\includegraphics[scale=0.7]{Figures/BP4.png}
			\end{figure}
	
	\section{Creditos}
	
		\textbf{Profª}: Gina Maira Barbosa de Oliveira
		
		\textbf{Aluno}: Gabriel Augusto Marson 
	
\end{document}