\documentclass[hidelinks,12pt]{article}

\usepackage[brazil]{babel}
\usepackage[utf8]{inputenc}
\usepackage{amsmath}
\usepackage{amsfonts}
\usepackage{amssymb}
\usepackage{indentfirst}
\usepackage{color}
\usepackage{mathrsfs}
\usepackage{pgfplots}
\usepackage{hyperref}
\usepackage{fancyhdr}
\usepackage[export]{adjustbox}
\newcommand{\icon}[1]{\includegraphics[height=12pt]{#1}}
\newcommand{\bigicon}[1]{\includegraphics[height=50pt]{#1}}

\newcommand{\iconb}[1]{\includegraphics[height=20pt]{#1}}
\setcounter{secnumdepth}{5}

\fancypagestyle{plain}{%
	\fancyfoot{}%
	\fancyhead{}%
}


\begin{document}
\pagenumbering{gobble}
\pagestyle{fancy}


\lhead{\bigicon{Figures/ufu}}
\chead{{\footnotesize UNIVERSIDADE FEDERAL DE UBERLÂNDIA \\ FACULDADE DE CIÊNCIA DA COMPUTAÇÃO \\ Inteligência Artificial} \\ \scriptsize{Av. João Naves de Ávila 2121, Campus Santa Mônica} }
\rhead{\bigicon{Figures/facom}}
\lfoot{}
\cfoot{}
\rfoot{}
\vspace*{8.5cm}
\begin{figure}[!h]
	\centering
	\Huge{Manual Referente ao Trabalho de A*}
\end{figure}

\vspace{5cm}
\noindent\textbf{Alunos:} Eduardo Costa, Frederico Franco, Gabriel Marson\\
\textbf{Profº.:} Carlos Lopes



\newpage
\fancyhead[C]{}
\fancyhead[R]{}
\fancyhead[L]{\leftmark}
\fancyfoot{}
\fancyfoot[L]{{\footnotesize  Inteligência Artificial}}
\fancyfoot[C]{\hspace{1.5cm}\thepage}
\fancyfoot[R]{{\footnotesize A* - Zelda}}
\pagenumbering{arabic}


{\let\thefootnote\relax\footnotetext{\textit{UFU, Universidade Federal de Uberlândia, Minas Gerais, Brasil}}}

\newpage

\section{Introdução}

	O relatório contém breves informações à respeito do trabalho pedido pelo professor bem como as formas de executá-lo.

\section{Especificação de Execução}
	
	
	\subsection{Geração do arquivo}
	
		O arquivo foi gerado pela IDE IntelliJ. Ele é um .jar que acompanha a pasta dos Maps a qual é imprescíndivel para a execução do programa pois os mapas do "jogo" estão contidos nessa pasta.
	
	\subsection{Execução}
		Ao baixar o arquivo, independentemente do sistema(Windows , Mac, Linux) , será necessário ter o Java JDK 1.8.0\underline{ }91  ou superior para executá-lo. Recomenda-se a execução pelo terminal por meio do comando \textbf{java -jar A.star.jar}. O programa será iniciado normalmente desde que esteja no mesmo diretório da pasta Maps.
	
	\subsection{Controles}
		Seguem os controles para visualizar o progressso do personagem no mapa.
		\begin{itemize}
			\item $\longrightarrow$ para ver os movimentos sucessores do caminho que será feito.
			\item $\longleftarrow$ para ver os movimentos antecessores do caminho feito.
		\end{itemize}
		
		
	
	
		
		
	
	
\end{document}